\documentclass{article}
\usepackage{listings}
\usepackage{color}

\definecolor{dkgreen}{rgb}{0,0.6,0}
\definecolor{gray}{rgb}{0.5,0.5,0.5}
\definecolor{mauve}{rgb}{0.58,0,0.82}

\lstset{frame=tb,
  language=Java,
  aboveskip=3mm,
  belowskip=3mm,
  showstringspaces=false,
  columns=flexible,
  basicstyle={\small\ttfamily},
  numbers=none,
  numberstyle=\tiny\color{gray},
  keywordstyle=\color{blue},
  commentstyle=\color{dkgreen},
  stringstyle=\color{mauve},
  breaklines=true,
  breakatwhitespace=true
  tabsize=3
}

\begin{document}
\title{Homework 4 CS-408}
\author{Huy Le }
\maketitle
\textbf{Question 9: } What are the advantages and disadvantage of keyword
parameter?\\
\textbf{Answer:} Advantages: parameters can appear in any order there by
avoiding parameter correspondence errors. Disadvantages: User must know the formal parameter's names\\
\\
\textbf{Question 16: } What are the modes, the conceptual models for transfer, the advantages, and the disadvantages of pass-by-value, pass-by-value-result, and pass-by-reference parameter passing methods?\\
\textbf{Answer: } The modes are 
\begin{itemize}
\item In mode
\item Out mode
\item Inout mode
\end{itemize}
\textbf{Pass-by-Value: } Advantages : for scalars it is fast, in both linnkage cost and access time. Disadvantages: additional storage is required (stored twice) and the actual move can be costly (for large parameter).\\
\textbf{Pass-by-Result} Advantages and disadvantages are like pass-by-value but with some more disadvantages which are actual parameter collision, and the spcification needed for when the address to be used to return the parameter value must be computed\\.
\textbf{Pass-by-Value-Result} Advantages: it is very fast for small size data. Disadvantages: those of pass-by-result, and those of pass-by-value.\\

\textbf{Pass-by-References} Advantages: passing process is efficient( no coypying and no duplicate storage). Disadvantages: slower accesses to formal parameters. Potential for unwanted side effects(collision). Unwanted Aliases.\\

\textbf{Question 20: } Describing the problem of passing multidimensional array as parameters\\
\textbf{Answer: } Problems of passing multidimensional array are\\
\begin{itemize}
\item building a mapping function
\item the compiler needs to know the size of the array. 
\item in language use the row major ordering for matrices, the collumb width is required, this method doesn't allow matrixes with different numbers of collumns.
\end{itemize}

\textbf{Question 24:} What is an overloaded sub-program?\\
\textbf{Answer: } An overloaded subprogram is a sub-program with the same referencing environment sa another, generally different data type for parameters.\\ 

\textbf{Question 25: } What is parametric polymorphism \\
\textbf{Answer: } Parametric polymorphism is also known as subtype polymorphism or generics, and allows subprogram to take parameters of different types on different activation\\

\textbf{Question 27: } In what fundamental ways do the generic parameters to a Java 5.0 generic method differ from those of C++ methods?
\textbf{Answer: } 
\begin{itemize}
\item Generic parameters in Java must be classes
\item Restrictions can be specified on the range of classes can be passed to the generic method as generic parameter
\item Wildcard type of generic parameters 
\end{itemize}

\textbf{Question 5}\\
\begin{lstlisting}
void swap (int a, int b ){
	temp = a;
	a = b;
	b = temp;
}
void main()
{
	int value =2;
	list[5] = { 1,3,5,7,9};
	swap(value, list[0]);
	swap(list[0],list[1]);
	swap(value, list[value]); 
}
\end{lstlisting}
\textbf{Answer: }\\
-Pass by Value: \\
\indent { 1,3,5,7,9}\\
\indent { 1,3,5,7,9}\\
\indent { 1,3,5,7,9}\\
-Pass by Reference: \\
\indent { 2,3,5,7,9}\\
\indent { 3,2,5,7,9}\\
\indent { 3,1,5,7,9}\\
-Pass by Reference: \\
\indent { 2,3,5,7,9}\\
\indent { 3,2,5,7,9}\\
\indent { 3,2,1,7,9}\\

\textbf{question 7}
\begin{lstlisting}
void fun(int first, int second)
{
	first +=first;
	second += second;
}
void main()
{
	int list[2] = {1,3};
	fun(list[0],list[1]);
}
\end{lstlisting}
\textbf{Answer:}\\
-Pass by value: {1,3}\\
-Pass by Reference: {2,6}\\
-Pass by Reference: {2,6}\\
\indent {}
\end{document}