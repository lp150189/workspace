\documentclass{article}
\begin{document}
\title{History Reading Assignment Pages: 644-672 }
\author{Huy Le}
\maketitle
\section{Mainpoints in the reading}
\begin{itemize}
\item FDr
\item How the president deal with the banking crisis in 1930
\item CIO
\item Huey Long
\item WPA: Work Progress Administration
\item The social security Act in 1935.
\item Election in 1936
\item The new deal in American History
\end{itemize}

\section {Mainpoints explained}
\subsection{FDR}
- as he like to be called, born in 1882, fith cousin of Theodore Roosevelt. \\
- serving in the navy during World war I, he lost the use of his leg in 1921, he contracted polio.
- His speech in 1932 accepting the Democratic nomination for president. He promised the new deal to people\\
- He advocated a balanced federal budget and critized his opponent, Hoover for excessive government spending. \\
- Batter by the economic crisis, Americans in 1932 were desperate for new leader ship, and Rosevelt won a resounding victory.\\
- Brandeis (his secretary ) beileve that large corporation needed to be broken
up into many pieces. But the \textbf{brainstrust: } a group of academics that
included a number of Columbia University Professors doesn't agree, instead we
should regulate them.

\subsection{ How the president deal with the banking crisis in 1930}
- March 4, 1933. Rosevelt confronted a banking system on the verge of collapse.\\
- Bank funds invested in the stock market lost their value and panicked depositors withdrew their savings, bank after bank closed.\\
- Rosevelt declare the bank holiday to temporarily halting all bank operation, and called Congress into special session. \\
- March 9, rushed to pass textbf{the Emergency Banking Act} which provide  funds to shore up threatened institutions.\\
- \textbf{Glass Steagal Act: } barred commercial banks from becoming involved in buying and selling stocks\\
- \textbf{ Federal Deposit Insurance (FDIC)} a government system that insured the accounts of individual depositors.\\
- \textbf{National Industrial Recovery Act: } was to large extent modeled on the
goverment-business partnership establish ed by the War Industries Board of World War I. Establish the NRA ( National Recovery Administration.). the NRA established codes that set standard for productions, prices and wages in many industry but at the same time, Large companied dominated the the code-writing process. So NRA produce neither economic recovery or peace between workers and employers. \\
- \textbf{Federal Emergency Relief Administration} created in May 1933, to make grants to local agencies that aided those impoverished by the Depression. But they are much prefered to create temporary jobs, thereby combating unemployement while improving the nation's infrastructure of roads, bridges, public buildings, parks.\\
- \textbf{CCC:} Civilian Conservation Corps, set unemployed young men to work on projects like forest preservation, flood control, and the improvement of national parks, and wildlife preserve.\\
- \textbf{PWA} Public Works Administration. \$ 3.3 billions, built roads, schools, hospitals, and other public facilities.\\
- \textbf{ CWA} Civil Work Administration by January 1934, employed more than 4 million people in the contructions of high ways, tunnels, courthouses, and airports\\ 
- \textbf{TVA} Tennessee Valley Authority. built a series of dams to provide cheap electricity for homes and factories. For the first time, TVA put federal goverment to compete selling electricity with private companies.\\
- \textbf{AAA} Agriculture Adjustment Act. setting production quotas for major crops and paying farmers not to plant more. \\
- \textbf{ Home Owner Loan Corporations And Federal Housing Administration: } insured millions of long-term mortgages issued by private banks. It built thousands of units of low-rent housing. It became cheaper for most Americans to buy single family homes than to rent apartments.

\subsection{CIO}
- stands for Congres of Industrial Organization (CIO)s\\
- it aims nothing less than to secure economic freedom and industrial democracy for American workers.\\
- in 1936, United Auto workers AWU , a fledging CIO union. Rather than walking out of the Fisher Body Plant in Cleverland. thus enabling management to bring in strike breakers, workers halted production but remained inside. Sit-downs soon spread to GM plants in Flint, Michiga, the nerve center of automobile production, Demonstrating a remarkable spirit of unity, the striker cleaned the plant oiled the idle machinery, and settled disputes among themselves.\\
- The sit down campain even spread to the steel industry. \\
- The CIO put forward an ambitious program for federal action to shield Americans from economic and social insecurity, including housing, universal health care, and unemployment and old age insurance.\\


\subsection{Huey Long }
- national prominence .\\
- Took governors in 1928 and 1930 in the U.S senate.\\
- He used his dictatorial power to build roads, schools, and hospitals and to increase the tax burden on Louisiana's oil commpany.\\
- 1934, created 'Share our wealth movement' with the slogan ' Every man a King.' called for the confiscation of most of the wealth of the richese American, in order for a imediate grant of \$5,000 and a guaranteed job and annual income for all citizens.\\
- He was on the verge of annoucing a run for president when the son of a defeated Politician rival assasinnated him in 1935.\\

\subsection {WPA: Work Progress Administration}:
- created by Rosevelt, hired 3 million Americans, each year until 1934\\
- constructed 500,000 miles of roads, 600 airports, thousands of public building and bridges. It hired many out-of-work white collar wokers and professionals even doctors and dentists.\\
- The WpPA set hundreds of artist to work decorating public buildings with murals. \\
- hired writers to produce local histories and guidebooks to the forty-eight states and to record the recollections of oridnary Americans, including hundreds of former slaves.\\

\subsection{The social security Act in 1935.}
- unemployment insurance, old age pensions, aid to the disabled, the elderly poor, and families with dependent children.\\
- What is new is the American goverment would now supervise not simply temporary relief but a permanent system of social insurance.\\
- launched version of the welfare state- a term that originated in Britain during WWII refer to a system of income assistance, health coverage, and social service.\\
- Social security at first exclude large number of Americans like farmers, unmarried women, and non-whites.\\

\subsection {Election in 1936}
- Rooselvelt got supported bby working-class voters, and Republicans Kansas governor Alfred Landon got supported by businesses large.\\
- Roosevelt accpeting renomination, Rosevelt launch a blistering attack against economic royalist and said poor men are not free men.\\
- Landon denounced Social Security and other measures as threats to individual liberty.\\
- In the end 1936, Roosevelt won the election more than 60\% of the popular vote.\\
- \textbf{ New Deal Coalition :} the group of unite southern white and northern black, Protestant farmers and urban Catholic and Jewish ethnics, Industrial workers and middle-class home owners.\\

\subsection{ The new deal in American History}
- Compare with European wllfare states, Social security remainded restricted in scope and modest in cost.\\
- fail to address the problem of racial inequality, which in some ways it actually worsened.\\
- greatly expanded the federal goverment's role in relations between industry and labor\\
- Governemnt tells farmers what they cold and couldn't plan, required employer to deal with unions, insured bank deposits, regulated the stock market. loaned money to home owners, and provided payments to a majority of the elderly and unemployed. \\
- help to get electricity for rural area. construct public facilities.\\
- restore faith in government and democracy\\
- More than 15\% of workforce still remained unemployed in 1940\\
\end{document}