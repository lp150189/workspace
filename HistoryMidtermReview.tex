\documentclass{article}
\begin{document}
\title{ History Midterm Review}
\author{Huy Le}
\maketitle
\section{Keypoints on the test}
\subsection{ Henry George}
- 1839-1897\\
- The writer of the book "Progress and Poverty" (1879)\\
- George suggests that there is something radically wrong in the present social organization\\
- His book demanded more public attention than any book in economics in American history.\\
- His solution was "single tax" which would replace other taxes with a levy on increases in the value of real estate.
- In the labor campaign took place in Newyork in 1886, it was even surprised to him that he found himself thrust into the role of labor's candidate for mayor.\\
- His ain in running was bring attention to the single tax on land.\\
- But labor leaders wants to stop the courts from barring strikes and jailing unionists for conspiracy\\
- After the campaign, there were 70,000 voted for George, and he finished second, surpass even Republican candidate, Theodore Roosevelt.\\
\subsection{ Garrison Frazier}
- 1797\\
- voices of freedom.\\
- A Baptist minister chosen as the group's spokesman, what is slavery? responded that it meant one person's "receiving by irresistible power the work of another man and not by his consent." \\
- Freedom was defined by him as "placing us where we could reap the fruit of our own labor, and take care of ourselves." \\
- The way to accomplish this wa "to have land, and turn it and till it by our own labors".\\
- When he was asked that do you think that there is intelligence enough among slaves of the South to maintain themselves under the goverment of the United States. and \textbf{he just answer very shortly but also very decisively that "I think there is sufficient intelligence among us to do so"}
\subsection{ Social Gospel}
- the idea of Social Gospel wa s that freedom and spiritual self-development required an equalization of wealth and power.\\
- The social gospel movement originated as an effort to reform Protestant churches by expanding their appeal in poor urban neighberhood and create more attraction to people about the era's social ills like social inequality, crime, racial tensions, slums, child labors, inadequate labor unions, danger of war.\\
- they created missions and relief program to accomplish those tasks.\\
- Also a group of Catholic priest and bishop tried to alter Church's traditional value to movements for social reforms, and they also try to change the isolation of churches from the social problems. They think that church should help society to solve social ills.
\subsection{ Knights of Labor}
- one of the first labor organization, led by Terence V. Powderly.\\
- first group tried to organize unskilled workers as well as skilled, women alongside men, and blacks as well as whites.\\
- The group reached membership of 800,000 in 1886 \\
- got involved in a lot of strikes, political action, educational and social activities.\\
\subsection{ Black Codes}
\subsection{ Booker T. Washington}
\subsection{ Chinese exclusion Act}
\subsection{ Red Scare}
\subsection{ Nigara Movement}
\subsection{ Andrew Carnegie}
\subsection{ Treaty of Versailles}
\subsection{ F.D Roosevelt}
\subsection{ Eugen Debs }
\subsection{ The Second Industrial Revolution}
\subsection{ Progressivism}
\subsection{ Social Darwinism}
\subsection{ The Muckrakers}
\subsection{ Theodore Roosevelt}
\subsection{ The Great Depression}
\subsection{ Prohibition}
\subsection{ The Poppulist Party}
\subsection{ John Marshal Harlan}

\section{ Essay questions}
\textbf{Question 1:} \textit{Explain the historical background and the various debates surrounding the Treaty of Versailles. Why did the senate ultimately refuse to ratify the treaty or join the League of Nations}\\
\textbf{Answer:}       \\

\textbf{Question 2:} \textit{What were the main characteristics of the Progressive  era ? How did each president of the progressive era view the role of the federal government?}\\
\textbf{Answer:}       \\


\textbf{Question 3:} \textit{Analyze the various roles women played during the Progressive era, from
social reformer, to feminist, to suffragette. How did various women define freedom?}\\
\textbf{Answer:}       \\


\textbf{Question 4:} \textit{Fully discuss the successes and failures of Reconstruction. Be sure to explain how freedom was expanded or constricted for various groups of people.}\\
\textbf{Answer:}       \\



\textbf{Question 5:} \textit{What is the Great Depression? How did Hoover and F.D.Roosevelt response to the Great Depression? Also Please focus on the “the first and second New Deal"}\\
\textbf{Answer:}       \\


\section{Notes in Class Midterm}
- Each keyterm could be answered in a paragraph\\
- 



\end{document}